\documentclass[12pt,twoside,draft]{cmpart}
\usepackage[T1,T2A]{fontenc}

\renewcommand{\quotesinglbase}{,}

% Description of some options:
% draft      - draft version of the document
% bysec      - propositions, lemmas, definitions etc (not theorems)
%              are numbered within each \section
% thmbysec   - theorems are numbered within each \section
% win        - CP-1251 encoding (Windows)
% koi        - KOI8_U encoding (Unix)
%
%  The following packets already included by cmparticle class:
% amsmath,amsfonts,amsthm,amssymb,latexsym
%
% Theorem-like environments already defined by cmparticle class:
% {theorem}     - Theorem
% {proposition} - Proposition
% {lemma}       - Lemma
% {corollary}   - Corollary
% {remark}      - Remark
% {definition}  - Definition
% {example}     - Example
% {examples}    - Examples
% as well as their analogs with star (for unnumbered environments):
% {theorem*}, {proposition*} and so on.


% YOUR DEFINITIONS HERE ------------------------------

% ----------------------------------------------------


\author{Author1 A.A., Author2 B.B.}
\title{Title of the article in English}

% The following command should be used ONLY if
% title of the article is too long for colontitles
\shorttitle{Short title of the article}

\institute{Univeristy, Address, City, Country (Author1 A.A.)\\
Univeristy, Address, City, Country (Author2 B.B.)}

\email{mail1@domain.ua (Author1 A.A.), mail2@domain.ua (Author2 B.B.)}

\enabstract{Author1 A.A., Author2 B.B.}%
{Title of the article in English}%
{An abstract in English. An abstract in English. An abstract in English.
An abstract in English. An abstract in English. An abstract in English.

An abstract in English. An abstract in English. An abstract in English.
An abstract in English. An abstract in English. An abstract in English.}
{keyword1, keyword2, phrase1, phrase2}

\uaabstract{Автор1 А.А., Автор2 Б.Б.}%
{Назва статті українською}%
{Анотація українською мовою. Анотація українською мовою. Анотація українською мовою.
Анотація українською мовою. Анотація українською мовою.}
{слово1, слово2, фраза1, фраза2}

\thanks{Authors want to thanks \dots}

\subjclass{26E40, 68T15, 971Ixx, 97N60, 97N80, 97P40, 97R20}
% 2010 Mathematics Subject Classification
% According to http://www.ams.org/msc/msc2010.html


\UDC{004.4, 004.6, 004.9, 510.21, 510.24, 510.25, 510.6, 519.68}
% Universal Decimal Classification (УДК)
% It will be indicated by Editorial Team
% (for those authors, who don't understand what is it)

\received{04.02.2021}  % article received date
% \revised{05.02.2021} % article revised date

\def\baselinestretch{1.1}

\begin{document}

\maketitle

%%% ----------------------------------------------------------------------

\section*{Introduction}
Lorem ipsum dolor sit amet, consectetur adipiscing elit, sed do eiusmod tempor 
incididunt ut labore et dolore magna aliqua. Ut enim ad minim veniam, 
quis nostrud exercitation ullamco laboris nisi ut aliquip ex ea commodo consequat. 
Duis aute irure dolor in reprehenderit in voluptate velit esse cillum dolore eu fugiat nulla pariatur. 
Excepteur sint occaecat cupidatat non proident, sunt in culpa qui officia deserunt mollit anim id est laborum.

\section{First section}
Lorem ipsum dolor sit amet, consectetur adipiscing elit, sed do eiusmod tempor 
incididunt ut labore et dolore magna aliqua. Ut enim ad minim veniam, 
quis nostrud exercitation ullamco laboris nisi ut aliquip ex ea commodo consequat. 
Duis aute irure dolor in reprehenderit in voluptate velit esse cillum dolore eu fugiat nulla pariatur. 
Excepteur sint occaecat cupidatat non proident, sunt in culpa qui officia deserunt mollit anim id est laborum.

\subsection{Subsection of the First Section}
Lorem ipsum dolor sit amet, consectetur adipiscing elit, sed do eiusmod tempor 
incididunt ut labore et dolore magna aliqua. Ut enim ad minim veniam, 
quis nostrud exercitation ullamco laboris nisi ut aliquip ex ea commodo consequat. 
Duis aute irure dolor in reprehenderit in voluptate velit esse cillum dolore eu fugiat nulla pariatur. 
Excepteur sint occaecat cupidatat non proident, sunt in culpa qui officia deserunt mollit anim id est laborum.

\subsubsection{Subsubsection of the subsection of the First Section}
Lorem ipsum dolor sit amet, consectetur adipiscing elit, sed do eiusmod tempor 
incididunt ut labore et dolore magna aliqua. Ut enim ad minim veniam, 
quis nostrud exercitation ullamco laboris nisi ut aliquip ex ea commodo consequat. 
Duis aute irure dolor in reprehenderit in voluptate velit esse cillum dolore eu fugiat nulla pariatur. 
Excepteur sint occaecat cupidatat non proident, sunt in culpa qui officia deserunt mollit anim id est laborum.

\paragraph{Some paragraph may start like this one.}
Lorem ipsum dolor sit amet, consectetur adipiscing elit, sed do eiusmod tempor 
incididunt ut labore et dolore magna aliqua. Ut enim ad minim veniam, 
quis nostrud exercitation ullamco laboris nisi ut aliquip ex ea commodo consequat. 
Duis aute irure dolor in reprehenderit in voluptate velit esse cillum dolore eu fugiat nulla pariatur. 
Excepteur sint occaecat cupidatat non proident, sunt in culpa qui officia deserunt mollit anim id est laborum.

\begin{definition}
Definition of something.
\end{definition}

\begin{proposition*}
Unnumbered proposition.
\end{proposition*}

In analogous way an author can produce another environments like
Theorem, Lemma, Definition etc.

To cite some publication(s) an author should use the command 
\verb"\cite{k1,k2,k3}" to produce this \cite{k1,k2,k3}.
Using the direct numbers in square brackets is prohibited!

\begin{theorem}
This is a numbered theorem with inline formula $E=mc^2$.
\end{theorem}
\begin{proof}
We are proving the theorem. We are proving the theorem. We are proving the theorem.
We are proving the theorem. We are proving the theorem. We are proving the theorem.

We are proving the theorem. We are proving the theorem. We are proving the theorem. 
Thus, the theorem is proved. 
\end{proof}

The displayed equations may be presented in two versions: numbered and unnumbered.
Note, that an unreferenced equation should be unnumbered!

The numbered equation is the following
\begin{equation}\label{eq1}
 e^x=\sum_{n=0}^\infty \frac{x^n}{n!},
\end{equation}
and unnumbered one is
\[
 \int_a^b f(x)\,dx=F(b)-F(a).
\]

To cite a numbered equation an author should use the command \verb"\eqref" to 
produce this \eqref{eq1}. Using the direct numbers in round brackets is prohibited!

In multiline equations, signs like 
$=$, $\le$, $<$, $+$, $-$ etc should be in next row, see the example:
\[
\begin{split}
\iint\limits_{R_1\le x^2+y^2\le R_2} e^{-x^2-y^2}\,dx\,dy
=&\int_0^{2\pi} d\varphi\int_{R_1}^{R_2} e^{-r^2} r\,dr\\
=&\pi\big(-e^{-r^2}\big)\big\vert_{R_1}^{R_2}=\pi\big(e^{-R_1^2}-e^{-R_2^2}\big).
\end{split}
\]

\begin{corollary}
The text of a corollary.
\end{corollary}

\begin{lemma}
The text of a lemma.
\end{lemma}

\begin{proposition}
The text of a proposition.
\end{proposition}

\begin{remark}
The Text of a remark.
\end{remark}

\begin{example}
Some example is presented here.
\end{example}

\section{Formatting of the reference list}

All references should be cited within the text; 
otherwise, these references should be removed from the reference list
(see position 8 in reference list).

References must be listed in alphabetical order. 
The titles of journals should be abbreviated to the style used by AMS 
(http://www.ams.org/msnhtml/serials.pdf).

If the referred paper has a Digital Object Identifier (DOI)
it should be indicated.

The following reference style should be used
(positions \cite{k1,k2,k3,k4,k5,k6,k7} are some abstract examples,
and concrete examples are presented in positions \cite{ke1,ke2,ke3,ke4,ke5,ke6,ke7}).

\begin{thebibliography}{999}

% *** A BOOK (in English)
\bibitem{k1} Author1 A.A., Author2 B.B., Author3 C.C. Title of the book. PublishingHouse, City, Year.

% *** A BOOK as a part of Series (in English)
\bibitem{k2} Author1 A.A., Author2 B.B., Author3 C.C. Title of the book. In: Editor1 A.A.,
Editor2 B.B. (Eds.)
SeriesTitle, Number. PublishingHouse, City, Year.

% *** A BOOK (in other language)
\bibitem{k3} Author1 A.A., Author2 B.B., Author3 C.C. English translation of title of the
book.
PublishingHouse, City, Year. (in Language)

% *** AN ARTICLE (in English)
\bibitem{k4} Author1 A.A., Author2 B.B., Author3 C.C. \emph{Title of the article}.
Title of the Journal Year, \textbf{Volume} (Number), PageF--PageL.
doi: N/A

% *** AN ARTICLE (in other language)
\bibitem{k5} Author1 A.A., Author2 B.B., Author3 C.C. \emph{English translation of title
of the article}. Title of the Journal Year, \textbf{Volume} (Number), PageF--PageL.
doi: N/A (in Language)

% *** A BOOK OF ABSTRACTS
\bibitem{k6} Author1 A.A., Author2 B.B. Title of abstract. In: Editor1 A.A., Editor2 B.B.
(Eds.) Proc. of the Intern. Conf.
``Title of the Conference'', City, Country, Month DateF--DateL, Year, PublishingHouse, City, Year, PageF--PageL.

% *** AN ARTICLE, translated in English from other language
\bibitem{k7} Author1 A.A., Author2 B.B., Author3 C.C. \emph {Title of the article}.
Title of the Journal Year, \textbf{Volume} (Number), PageF--PageL. doi: N/A
(translation of Title of the Journal Year, \textbf{Volume} (Number),
PageF--PageL. doi: N/A (in Language))

%%%%%%%%%%%%%%%%%%%%%%%%%%%%%

\bibitem{ke8} Author A.A. \textit{En example of uncited article}. Journal 2021, \textbf{1} (1), 1--2.

%%%%%%%%%%%%%%%%%%%%%%%%%%%%%

% *** A BOOK (in English)
\bibitem{ke1} Dineen S. Complex analysis on infinite-dimensional spaces.
 Springer-Verlag, London, 1999.

% *** A BOOK as a part of Series (in English)
\bibitem{ke2} Defant A., Floret K. Tensor norms and operator ideals.
In: Nachbin L. (Ed.) Mathematics Studies, 176.
North-Holland, Amsterdam, 1993.

% *** A BOOK (in other language)
\bibitem{ke3} Kadets V.M. A course of functional analysis and measure theory.
Chyslo, Lviv, 2012. (in Ukrainian)

% *** AN ARTICLE (in English)
\bibitem{ke4} Lopushansky O., Zagorodnyuk A. \emph{Representing measures
and infinite-dimensional holomorphy}. J. Math. Anal. \& App. 2007, \textbf{333} (2), 614--625.
doi:10.1016/j.jmaa.2006.09.035

% *** AN ARTICLE (in other language)
\bibitem{ke5} Malytska G.P. \emph{Systems of equations of Kolmogorov type.}
Ukr. Math. J. 2009, \textbf{12} (3), 1650--1563. (in Ukrainian)

% *** A BOOK OF ABSTRACTS
\bibitem{ke6} Sharyn S.V. Generalized Fourier transformation of
polynomial ultradistributions. In: Proc. of the Intern. Sci. Conf. ``Infinite
Dimensional Analysis and Topology'', Ivano-Frankivsk, Ukraine,
May 27--June 1, 2009. Ter. Druku, Ivano-Frankivsk, 2009, 134--135.

% *** AN ARTICLE, translated in English from other language
\bibitem{ke7} Zagorodnuyk A.V., Mitrofanov M.A. \emph {An analog of Wiener’s theorem for
infinite-dimensional Banach spaces}. Math. Notes 2015, \textbf{97} (1-2), 179--189.
doi:10.1134/S0001434615010204 (translation of Mat. Zametki 2015, 97 (2), 191--202.

\end{thebibliography}

\end{document}
