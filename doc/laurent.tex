\documentclass[12pt,twoside,draft]{cmpart}
\usepackage[T1,T2A]{fontenc}
\usepackage[utf8]{inputenc}
\usepackage{hyperref}
\usepackage{newunicodechar}
\usepackage{verbatim}

\makeatletter
\renewcommand{\verbatim@font}{\ttfamily\small} % Change font here
\makeatother

\newunicodechar{λ}{$\lambda$}
\newunicodechar{Π}{$\Pi$}
\newunicodechar{Σ}{$\Sigma$}
\newunicodechar{ε}{$\epsilon$}
\renewcommand{\quotesinglbase}{,}


% Description of some options:
% draft      - draft version of the document
% bysec      - propositions, lemmas, definitions etc (not theorems)
%              are numbered within each \section
% thmbysec   - theorems are numbered within each \section
% win        - CP-1251 encoding (Windows)
% koi        - KOI8_U encoding (Unix)
% utf8       - UTF-8 encoding (Unix)
%
%  The following packets already included by cmparticle class:
% amsmath,amsfonts,amsthm,amssymb,latexsym
%
% Theorem-like environments already defined by cmparticle class:
% {theorem}     - Theorem
% {proposition} - Proposition
% {lemma}       - Lemma
% {corollary}   - Corollary
% {remark}      - Remark
% {definition}  - Definition
% {example}     - Example
% {examples}    - Examples
% as well as their analogs with star (for unnumbered environments):
% {theorem*}, {proposition*} and so on.

% YOUR DEFINITIONS HERE ------------------------------

% ----------------------------------------------------

\author{Sokhatskyi M.E.}
\title{Laurent: Syntentic Analysis}

\shorttitle{}

\institute{Groupoid Infinity, Laboratory of National Technical University of Ukraine, м. Київ, Україна, 03056,
  Політехнічна вул. 14-А, корпус 14 (прикладної математики)}

\email{cmp@5ht.co (Sokhatskyi M.E.)}

\enabstract{Sokhatskyi M.E.}
{Laurent proof assistant}
{The formalization of mathematical analysis in proof assistants has seen significant
 advancements with systems like Lean and Coq, which have successfully mechanized key
 results in functional analysis, including Bochner integration, $L^2$ spaces, theory of distributions.
 This article introduces Laurent, a novel proof assistant designed to bridge
 classical and constructive analysis through a unique architecture. Unlike
 general-purpose systems, Laurent embeds explicit primitives for calculus
 and measure theory directly into its core, complemented by a tactics language
 inspired by Lean, Coq, and its recent near tactics. We present its foundational
 constructs, demonstrate its application to theorems in sequences, integration,
 and $L^2$ spaces, and argue that its design offers a more intuitive mechanization
 of analysis, aligning with the mathematical reasoning of classical analysts while
 retaining constructive rigor. This work positions Laurent as a specialized tool
 for functional analysis within the growing domain of computational mathematics.}
{Type Theory, Functional Analysis, Computational Mathematics}

\uaabstract{Sokhatskyi M.E.}
{Система доведення теорем Laurent}
{Формалізація математичного аналізу в системах доведення теорем досягла значних успіхів завдяки таким системам,
 як Lean і Coq, які успішно механізували результати аналізу, таків як інтеграл Соломона Бохнера, простори $L^2$, теорія розподілів Лорана Шварца.
У цій статті представлено Laurent — нова система доведення теорем, розроблена для поєднання класичного та конструктивного
аналізу. На відміну від систем загального призначення,
Laurent вбудовує явні примітиви для теорії міри, теорії множин та аналізу в своє ядро,
доповнене мовою тактик, натхненною Lean, Coq та його спеціалізованими тактиками для околів.
Ми описуємо її базові конструкції, демонструємо їх застосування до теорем про послідовності,
інтеграцію та простори $L^2$ і стверджуємо, що його дизайн пропонує більш інтуїтивну механізацію
аналізу, узгоджену з математичним мисленням класичних аналітиків, зберігаючи при цьому конструктивну
строгість в дусі Ерретта Бішопа (його формулювання теж виражаються).}
{Теорія типів, функціональний аналіз, обчислювальна математика}

\thanks{Sokhatskyi M.E. would like to acknowledge support by the Ministry of Health of Ukraine.}

\subjclass{26E40, 68T15, 68Q45,  }
\UDC{510.6, 519.68}
\received{08.05.2024}

\def\baselinestretch{1.1}

\begin{document}

\maketitle

%%% ----------------------------------------------------------------------

\section{Introduction}
The mechanization of mathematical theorems has transformed modern mathematics,
enabling rigorous verification of proofs through computational tools known as
proof assistants. Systems like Lean and Coq have emerged as leaders in this field,
leveraging dependent type theory to formalize a wide range of mathematical domains.

Despite their successes, Lean and Coq often rely on extensive libraries (e.g., Lean’s \texttt{mathlib}
or Coq’s Mathematical Components) and general-purpose tactics—such as \texttt{ring}, \texttt{field},
or \texttt{linearith}—that, while effective, can feel detached
from the intuitive reasoning of classical analysis. This gap has
inspired the development of Laurent, a proof assistant tailored
for mathematical analysis, functional analysis, and distribution theory.
Laurent integrates explicit primitives for sets, measures, and calculus
into its core, paired with a tactics language akin to Lean and Coq, augmented
by recent innovations like \texttt{near} tactics \cite{af18}. This design aims
to reflect the spirit of classical mathematics while enabling constructive
theorem-proving, offering a specialized tool for researchers in functional analysis.

This article outlines Laurent’s architecture and demonstrates its mechanization
of classical and constructive theorems, drawing on examples from sequences,
Lebesgue integration, and $L^2$ spaces. We target formal mathematics audience
emphasizing computational mathematics and frontier research in functional analysis.

$$
\begin{array}{c} \\
   \mathrm{Laurent} := \mathrm{MLTT}\ |\ \mathrm{CALC} \\
   \mathrm{MLTT} := \mathrm{Cosmos}\ |\ \mathrm{Var}\ |\ \mathrm{Forall}\ |\ \mathrm{Exists}\ \\
   \mathrm{CALC} := \mathrm{Base}\ |\ \mathrm{Set}\ |\ \mathrm{Q}\ |\ \mathrm{Mu}\ |\ \mathrm{Lim} \\
   \mathrm{Cosmos} := \mathbf{Prop}\ :\ \mathbf{U_0}\ :\ \mathbf{U_1} \\
   \mathrm{Var} := \mathbf{var}\ \mathrm{ident}\ |\ \mathbf{hole} \\
   \mathrm{Forall} := \forall\ \mathrm{ident}\ \mathrm{E}\ \mathrm{E}\ |\ \lambda\ \mathrm{ident}\ \mathrm{E}\ \mathrm{E}\ |\ \mathrm{E}\ \mathrm{E} \\
   \mathrm{Exists} := \exists\ \mathrm{ident}\ \mathrm{E}\ \mathrm{E}\ |\ (\mathrm{E}, \mathrm{E})\ |\ \mathrm{E}\mathbf{.1}\ |\ \mathrm{E}\mathbf{.2} \\
   \mathrm{Base} := \mathbb{N}\ |\ \mathbb{Z}\ |\ \mathbb{Q}\ |\ \mathbb{R}\ |\ \mathbb{C}\ |\ \mathbb{H}\ |\ \mathbb{O}\ |\ \mathbb{V}^n\ \\
   \mathrm{Set} := \mathbf{Set}\ |\ \mathbf{SeqEq}\ |\ \mathbf{And}\ |\ \mathbf{Or}\ 
                |\ \mathbf{Complement}\ |\ \mathbf{Intersect}\ \\
                |\ \mathbf{Power}\ |\ \mathbf{Closure}\ |\ \mathbf{Cardinal}\ \\
   \mathrm{Q} := \mathbf{-/\hspace{-1mm}\sim}\ |\ \mathbf{Quot}\ |\ \mathbf{Lift_Q}\ |\ \mathbf{Ind_Q} \\
   \mathrm{Mu} := \mathbf{mu}\ |\ \mathbf{Measure}\ 
                |\ \mathbf{Lebesgue}\ |\ \mathbf{Bochner}\ \\
   \mathrm{Lim} := \mathbf{Seq}\ |\ \mathbf{Sup}\ |\ \mathbf{Inf}\ 
               |\ \mathbf{Limit}\ |\ \mathbf{Sum}\ |\ \mathbf{Union}\ \\
         \\
\end{array}
$$

\section{Background: Lean and Coq in Functional Analysis}
Lean, developed by Leonardo de Moura, is built on a dependent type theory variant
of the Calculus of Inductive Constructions (CIC), with a small inference kernel
and strong automation. Its mathematical library, \texttt{mathlib},
includes formalizations of Lebesgue measure, Bochner integration, and $L^2$ spaces,
upporting proofs up to research-level mathematics.
Tactics like \texttt{norm\_num} and \texttt{continuity} automate routine steps,
though their generality can obscure domain-specific insights.

Both systems, while powerful, prioritize generality over domain-specific efficiency \cite{bo14}.
Laurent addresses this by embedding analysis primitives directly into its core,
inspired by recent advancements in near tactics, which enhance proof search with
contextual awareness.

\section{The Laurent Theorem Prover}
Laurent is designed to mechanize theorems in classical and constructive analysis
with a focus on functional analysis. Its core is built on dependent types—Pi (functions)
and Sigma (pairs)—augmented by explicit primitives for sets, measures, and calculus
operations. Unlike Lean and Coq, where such notions are library-defined, Laurent’s
primitives are native, reducing abstraction overhead and aligning with classical
mathematical notation.

\subsection{Basic Constructs and Set Theory}
Laurent’s syntax begins with fundamental types: natural numbers ($\mathbb{N}$),
integers ($\mathbb{Z}$), rationals ($\mathbb{Q}$), reals ($\mathbb{R}$),
complex numbers ($\mathbb{C}$), quaternions ($\mathbb{H}$),
octanions ($\mathbb{O}$) and $n$-vectors ($\mathbb{V}^n$)
all embedded in the core. Sets are first-class objects, defined using lambda abstractions. For example:
\begin{verbatim}
let set_a : exp =
  Set (Lam ("x", Real,
    RealIneq (Gt, Var "x", Zero)))
\end{verbatim}
represents the set $\{ x : \mathbb{R} \mid x > 0 \}$. Operations like supremum and infimum are built-in:
\begin{align*}
    \sup \{ x > 0 \} &= +\infty, \\
    \inf \{ x > 0 \} &= 0,
\end{align*}
computed via \texttt{Sup set\_a} and \texttt{Inf set\_a}, reflecting the unbounded and bounded-below nature of the positive reals.

\subsection{Measure Theory and Integration}
Measure theory is central to functional analysis, and Laurent embeds Lebesgue measure as a primitive:
\begin{verbatim}
let interval_a_b (a : exp) (b : exp) : exp =
  Set (Lam ("x", Real,
   And (RealIneq (Lte, a, Var "x"),
        RealIneq (Lte, Var "x", b))))

let lebesgue_measure (a : exp) (b : exp) : exp =
  Mu (Real, Power (Set Real), Lam ("A", Set Real,
    If (RealIneq (Lte, a, b),
        RealOps (Minus, b, a),
        Infinity)))
\end{verbatim}
This defines $\mu([a, b]) = b - a$ for $a \leq b$, otherwise $\infty$. The Lebesgue integral is then constructed:
\begin{verbatim}
let integral_term : exp =
  Lam ("f", Forall ("x", Real, Real), Lam ("a", Real, Lam ("b", Real,
    Lebesgue (Var "f", Mu (Real, Power (Set Real), Lam ("A", Set Real,
      If (And (RealIneq (Lte, Var "a", Var "b"),
              SetEq (Var "A", interval_a_b (Var "a") (Var "b"))),
          RealOps (Minus, Var "b", Var "a"), Zero))),
      interval_a_b (Var "a") (Var "b")))))
\end{verbatim}
representing $\int_{[a,b]} f \, d\mu$, with type signature $f, a, b : \mathbb{R} \to \mathbb{R}$.

\subsection{$L^2$ Spaces}
The $L^2$ space, critical in functional analysis, is defined as:
\begin{verbatim}
let l2_space : exp =
  Lam ("f", Forall ("x", Real, Real),
    RealIneq (Lt,
      Lebesgue (Lam ("x", Real,
        RealOps (Pow, RealOps (Abs, App (Var "f", Var "x"), Zero),
        RealOps (Plus, One, One))),
        lebesgue_measure Zero Infinity, interval_a_b Zero Infinity),
      Infinity))
\end{verbatim}
This encodes $\{ f : \mathbb{R} \to \mathbb{R} \mid \int_0^\infty |f(x)|^2 \, d\mu < \infty \}$, leveraging Laurent’s measure and integration primitives.

\subsection{Sequences and Limits}
Laurent mechanizes classical convergence proofs efficiently. Consider the sequence $a_n = \frac{1}{n}$:
\begin{verbatim}
let sequence_a : exp =
  Lam ("n", Nat, RealOps (Div, One, NatToReal (Var "n")))

let limit_a : exp =
  Limit (Seq sequence_a, Infinity, Zero,
    Lam ("ε", Real, Lam ("p", RealIneq (Gt, Var "ε", Zero),
      Pair (RealOps (Div, One, Var "ε"),
        Lam ("n", Nat, Lam ("q", RealIneq (Gt, Var "n", Var "N"),
          RealIneq (Lt, RealOps (Abs,
          RealOps (Minus, App (sequence_a, Var "n"), Zero), Zero),
            Var "ε")))))))
\end{verbatim}
This proves $\lim_{n \to \infty} \frac{1}{n} = 0$, with $\forall \varepsilon > 0$, $\exists N = \frac{1}{\varepsilon}$ such that $n > N$ implies $\left| \frac{1}{n} \right| < \varepsilon$.

\section{Near Tactics and Constructive Reasoning}
Laurent adopts a tactics language inspired by Lean and Coq, enhanced by INRIA’s
near tactics \cite{af18}. These tactics prioritize contextual proof search,
enabling efficient automation of analysis-specific steps (e.g., limit computations
or integral bounds). Unlike Lean’s \texttt{limit} or Coq’s \texttt{auto}, \texttt{near}
tactics adapt to the mathematical structure, reducing user effort while preserving
constructive rigor. Combined with Laurent’s classical core, this facilitates a
hybrid approach, mechanizing theorems both classically and constructively.

\section{Examples of Theorem Mechanization}
Laurent’s design excels in mechanizing foundational theorems across differential calculus, integral calculus, and functional analysis. Below, we present a selection of classical results formalized in Laurent, showcasing its explicit primitives and constructive capabilities.

\subsection{Taylor’s Theorem with Remainder}
Taylor’s Theorem provides an approximation of a function near a point using its derivatives. If $f : \mathbb{R} \to \mathbb{R}$ is $n$-times differentiable at $a$, then:
\[
f(x) = \sum_{k=0}^{n-1} \frac{f^{(k)}(a)}{k!} (x - a)^k + R_n(x),
\]
where $R_n(x) = o((x - a)^{n-1})$ as $x \to a$. 

In Laurent this encodes the theorem’s structure, with diff\_k
representing the $k$-th derivative and `remainder` satisfying
the little-o condition, verifiable via Laurent’s limit primitives.

\subsection{Fundamental Theorem of Calculus}
The Fundamental Theorem of Calculus links differentiation and integration.
If $f$ is continuous on $[a, b]$, then $F(x) = \int_a^x f(t) \, dt$ is
differentiable, and $F'(x) = f(x)$:

Laurent’s `Lebesgue` primitive and `diff` operator directly capture the integral and derivative, aligning with classical intuition.

\subsection{Lebesgue Dominated Convergence Theorem}
In functional analysis, the Dominated Convergence Theorem ensures integral convergence under domination. If $f_n \to f$ almost everywhere, $|f_n| \leq g$, and $\int g < \infty$, then $\int f_n \to \int f$:
This leverages Laurent’s sequence and measure primitives, with `Limit` automating convergence proofs via near tactics.

\subsection{Schwartz Kernel Theorem}
For distributions, the Schwartz Kernel Theorem states that every continuous bilinear form $B : \mathcal{D}(\mathbb{R}^n) \times \mathcal{D}(\mathbb{R}^m) \to \mathbb{R}$ is represented by a distribution $K \in \mathcal{D}'(\mathbb{R}^n \times \mathbb{R}^m)$ such that $B(\phi, \psi) = \langle K, \phi \otimes \psi \rangle$:
This uses Sigma types to pair the kernel $K$ with its defining property, reflecting Laurent’s support for advanced functional analysis.

\subsection{Banach Space Duality}
In Banach spaces, there’s a bijection between closed subspaces of $X$ and $X^*$ via annihilators: $A \mapsto A^\perp$, $B \mapsto {}^\perp B$. Laurent formalizes this as:
\begin{verbatim}
let bijection_theorem = Π (Set Real, ("X",
  If (banach_space (Var "X"),
    And (
      Π (Set (Var "X"), ("A",
        If (closed_subspace (Var "X", Var "A"),
            Id (Set (Var "X"), Var "A", pre_annihilator (Var "X",
                    annihilator (Var "X", Var "A"))), Bool))),
      Π (Set (dual_space (Var "X")), ("B",
        If (closed_subspace (dual_space (Var "X"), Var "B"),
            Id (Set (dual_space (Var "X")), Var "B", annihilator (Var "X",
                    pre_annihilator (Var "X", Var "B"))), Bool))))), Bool)))
\end{verbatim}
This showcases Laurent’s ability to handle normed spaces and duality, critical in functional analysis.

\subsection{Banach-Steinhaus Theorem}
The Banach-Steinhaus Theorem ensures uniform boundedness of operators.

If $\sup_{\alpha \in A} \|T_\alpha x\|_Y < \infty$ for all $x \in X$, then there exists $M$ such that $\|T_\alpha\|_{X \to Y} \leq M$:

This uses Laurent’s norm and operator primitives, with near tactics simplifying boundedness proofs.

\subsection{de Rham Theorem}
The de Rham Theorem relates differential forms and integrals over loops. For an open $\Omega \subset \mathbb{R}^n$ and a $C^1$ 1-form $\omega$, if $\int_\gamma \omega = 0$ for all loops $\gamma$, there exists $f$ such that $\omega = df$:
\begin{verbatim}
let de_rham_theorem =
  Π (Nat, ("n",   
    Π (Set (Vec (n, Real, RealOps RPlus, RealOps RMult)), ("Omega",
      Π (one_form Omega n, ("omega",
        And (c1_form Omega n (Var "omega"),
          And (Π (loop Omega n, ("gamma",
              Id (Real, integral (Var "omega", Var "gamma"), zero))),
            Σ (zero_form Omega, ("f", And (
                Id (one_form Omega n, Var "omega", differential (Var "f")),
                Π (Nat, ("m", If (cm_form Omega n (Var "m") (Var "omega"),
                  cm_form Omega n (Var "m") (Var "f"), Bool)))))))))))))
\end{verbatim}
This demonstrates Laurent’s capacity for topology and differential geometry, integrating forms and limits.

These examples highlight Laurent’s versatility, from basic calculus to advanced functional analysis, leveraging its native primitives and tactics for intuitive yet rigorous mechanization.



\section{Discussion and Future Directions}
Laurent has built-in primitives for streamline proofs in measure theory, integration, and $L^2$ spaces,
while its tactics language ensures flexibility. Compared to Lean’s library-heavy approach
or Coq’s constructive focus, Laurent balances classical intuition with formal precision,
making it accessible to analysts accustomed to paper-based reasoning. Future work includes
expanding Laurent’s tactics repertoire, formalizing advanced
theorems (e.g., dominated convergence, distribution theory).

Hosted at \footnote{\url{https://github.com/groupoid/laurent}}, Laurent invites community
contributions to refine its role in computational mathematics.

\section{Conclusion}
Laurent represents a specialized advancement in theorem mechanization, tailored for classical and constructive analysis.
By embedding analysis primitives and leveraging near tactics, it offers a unique tool for functional analysts,
complementing the broader capabilities of Lean and Coq. This work underscores the potential of domain-specific
proof assistants in advancing computational mathematics.

\begin{thebibliography}{999}


% *** AN ARTICLE (in English)
\bibitem[1]{af18} Affeldt R., Cohen C., Mahboubi A., Rouhling D.,  Strub P-Y. \emph{Classical Analysis with Coq},
Coq Workshop 2018, Oxford, UK doi:

% *** AN ARTICLE (in English)
\bibitem[2]{bo14} Boldo S., Lelay C., Melquiond G. \emph{Formalization of Real Analysis: A Survey of Proof Assistants and Libraries},
Mathematical Structures in Computer Science, 2016, 26 (7), pp.1196-1233. doi:10.1017/S0960129514000437

% *** A BOOK (in French)
\bibitem[3]{schwartz1967} Schwartz, L. \emph{Analyse Mathématique}, Hermann, Paris, 1967.

% *** A BOOK (in English)
\bibitem[4]{bishop1967} Bishop, E. \emph{Foundations of Constructive Analysis}, McGraw-Hill, New York, 1967.

% *** AN ARTICLE (in English)
\bibitem[5]{bridges1999} Bridges, D. \emph{Constructive Mathematics: A Foundation for Computable Analysis}, Theoretical Computer Science, 1999, 219 (1-2), pp.95--109.

% *** A PhD THESIS (in English)
\bibitem[6]{booij2020} Booij, A. \emph{Analysis in Univalent Type Theory}, PhD thesis, University of Birmingham, 2020. Available at: \url{https://etheses.bham.ac.uk/id/eprint/10411/7/Booij2020PhD.pdf}

% *** A PREPRINT (in English)
\bibitem[7]{murray2023} Murray, Z. \emph{Constructive Real Numbers in the Agda Proof Assistant}, 2023. Available at: \url{https://arxiv.org/pdf/2205.08354}

% *** A BOOK (in English)
\bibitem[8]{ziemer2017} Ziemer, W. P., Torres, M. \emph{Modern Real Analysis}, Springer, New York, 2017. Available at: \url{https://www.math.purdue.edu/~torresm/pubs/Modern-real-analysis.pdf}

\end{thebibliography}

\end{document}
